%! Author = phmatray
%! Date = 27/09/2024

% Preamble
\documentclass[12pt]{article}

% Packages
\usepackage{amsmath}
\usepackage{array}     % For better table formatting
\usepackage{geometry}  % For adjusting page geometry
\geometry{a4paper, margin=1in}  % Set margins

% Document
\begin{document}

    % Company Information
    \begin{flushright}
        \textbf{<<CompanyName>>} \\
        <<CompanyAddress>> \\
    \end{flushright}

    % Invoice Title
    \begin{flushleft}
        \textbf{Invoice Number:} <<InvoiceNumber>> \\
        \textbf{Date:} <<Date>> \\
        \textbf{Due Date:} <<DueDate>> \\
    \end{flushleft}

    \vspace{0.5cm}

    % Customer Information
    \begin{flushleft}
        \textbf{Bill To:} \\
        <<UserName>> \\
        <<CustomerAddress>> \\
    \end{flushleft}

    \vspace{0.5cm}

    % Invoice Table
    \begin{tabular}{| >{\raggedright}p{8cm} | >{\raggedleft}p{2.5cm} | >{\raggedleft}p{2.5cm} | >{\raggedleft\arraybackslash}p{3cm} |}
        \hline
        \textbf{Description} & \textbf{Unit Price (USD)} & \textbf{Quantity} & \textbf{Total (USD)} \\
        \hline
        <<for each item in Items>>  % Start of item loop
        <<item.Description>> & <<item.UnitPrice>> & <<item.Quantity>> & <<item.Total>> \\
        <<end for>>  % End of item loop
        \hline
        \textbf{Total Amount} & & & \textbf{<<TotalAmount>> USD} \\
        \hline
    \end{tabular}

    \vspace{1cm}

    % Footer
    \textbf{Thank you for your business!}

\end{document}
